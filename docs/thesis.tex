\documentclass[a4paper,12pt]{report}
	\usepackage[utf8]{inputenc}
	\usepackage{graphicx}
	\graphicspath{{figures/}}


\begin{document}
	\title{Virtual Private Networks}
	\author{Bogdan Ionescu}
	\date{\today}
	\maketitle
	\tableofcontents
	
	\chapter{A Brief Introduction to Networking}
	\section{What Is Networking?}
		In this day and age, networks are everywhere. Whenever we quickly search for something on the Internet, stream our favorite song on our phone or use an ATM, we rely on some kind of network. We can even turn off the lights or unlock our door with a simple voice command while comfortably sitting in our bed thanks to the IoT. Given the fact that the first message transmitted between two computers happened in 1959, it is astonishing how fast networking evolved - but what really is a network? 

		A \textit{network}, quite simply put, is a set of hardware devices connected together, either physically or logically, with the purpose of exchanging information. \textit{Networking} is the term that describes the processes involved in designing, implementing, managing, and otherwise working with networks. 
		
		Given the fact that networks have evolved at a very fast pace since 1959, there are many network technologies that are currently used. Although this allows for creating various kind of networks, it can sometimes be difficult to explain how this complex system works. The best way to understand networking is to break it down into several pieces and then decide what those pieces do and how they interact.
		
		In networking technologies, these pieces are almost always called \textit{layers}. Each layer contains hardware and/or software elements which work together to perform a particular task. Layers are conceptually arranged into a vertical \textit{stack}, called a \textit{network stack}, where each layer interacts only with the layer above and below it. The first layer, the one at the bottom, is responsible for concrete hardware tasks, such as hardware signaling and raw data transmission and, as you go up on the stack, the tasks performed get more and more oriented towards the software level.
		
		One crucial benefit of layering is that it allows technologies defined by different groups to interoperate. For this to be possible, it is necessary to define where each layer is in the network, what it is responsible of doing and how it can interact with others. This organization is called a \textit{networking model}.
		
	\section{The OSI Model}

\end{document}